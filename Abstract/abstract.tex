% ************************** Thesis Abstract *****************************
\begin{abstract}

%\hline
%\vspace{0.5cm}

% % add for hardbound version
% \subsubsection{Title}
% Using single-cell RNA-seq to assess the effect of common genetic variants on gene expression during development

% \subsubsection{Author}
% Anna Cuomo

% \subsubsection{Summary}
% Over the last fifteen years, genome-wide association studies (GWAS) have been used to identify thousands of DNA variants associated with complex traits and diseases, by exploiting naturally occurring genetic variation in large populations of individuals. 
% More recently, similar approaches have been applied to \gls{rnaseq} data to find variants associated with expression level, called expression quantitative trait loci (eQTL).
% Recent advances in experimental techniques have provided an unprecedented opportunity to measure gene expression at the single cell level, and the chance to study cellular heterogeneity. 
% This represents a remarkable advance over traditional bulk sequencing methods, particularly to study cell fate commitment events in development.
% The challenge of studying early human development is partially overcome by advances in stem cell technologies.
% In particular, induced pluripotent stem cells (iPSCs) and cells derived therefrom represent a fantastic system to study development \textit{in vitro}.
% In this thesis, I investigate the computational challenges of using single cell expression profiles to perform \gls{eqtl} mapping, and provide suitable approaches for the identification of cell type and context-specific eQTL using single cell expression profiles.
% I further explore the application of such methods across a range of human iPSC-derived cell types, using data from the \gls{hipsci} project. 

\par \gls{t2d}, a metabolic disorder, is associated with systemic inflammation, including that of pancreatic tissue and islets. In addition to this, the most critical determinant of \gls{t2d} is the progressive dysfunction of insulin secreting $\beta$-cells in the face of peripheral insulin resistance and impaired glucose tolerance. However, our understanding of the dynamics of islet immune cell infiltration during metabolic stresses such as overnutrition and aging is incomplete, as no focused analysis of immune cell signatures has been performed, and informative time-course data is missing. In addition, significant gaps also remain in our understanding of how $\beta$-cells respond to chronic metabolic stresses and transition from adaptive responses to $\beta$-cell dysfunction. Furthermore, there is a lack of consensus on the number of $\beta$-cell states during homeostasis and disease, underscoring the need to study $\beta$-cell plasticity, adaptation and failure in greater detail.\\

\par In this thesis, I weave together the field of pancreatic islet biology with single-cell approaches. Using previously established mouse models that mimic human \gls{t2d} pathogenesis, we reveal the heterogeneity and dynamic responses of immune cells and $\beta$-cells in response to several metabolic stresses.\\

\par In the first part of my thesis, we employ a multi-modal approach, combining imaging mass cytometry (\gls{imc}) and \gls{scr}, to investigate the impact of overnutrition and aging on immune cell dynamics within mouse pancreatic tissue and islets. We map the spatial distribution and transcriptional profiles of immune cells and reveal significant changes in the immune landscape, including the accelerated expansion of pro-inflammatory populations and a strongly altered inflammatory response during overnutrition compared to aging.\\

\par In the second part of my thesis, we construct a \gls{scr} atlas of $\beta$-cells across various mouse models to understand $\beta$-cell adaptation and dysfunction under increased workloads and hyperglycemia. We integrate in-house generated and previously published datasets in this meta-analysis and identify distinct $\beta$-cell subsets. We demonstrate that these $\beta$-cell subsets exist along a continuum corresponding to $\beta$-cell dysfunction and that $\beta$-cell dysfunction is associated with increased workload and loss of maturation.\\

\par Altogether, this dissertation provides valuable insights into the complexity of inflammatory responses and $\beta$-cell adaptation and failure in \gls{t2d} and sets a solid foundation for further studies aimed at mitigating the impact of metabolic diseases and improving clinical outcomes. 


% The dysfunction of insulin secreting $\beta$-cells and the inflammation of the pancreatic islets and tissue in the face of peripheral insulin resistance and impaired glucose tolerance eventually leads to the clinical onset of Type-2 diabetes. 

% Broadly speaking, this thesis weaves together the field of . The topic of pancreatic islets is a wide-ranging and highly researched area of biology, with insulin-producing $\beta$-cells at the core, which are responsible for maintaining blood glucose homeostasis in the body. . Single-cell transcriptomics, a transformative technology, has revolutionized biological research. By analyzing the transcriptome of individual cells, researchers can now routinely dissect the complexities of biological systems at an unprecedented cellular level. Powerful computational tools have emerged alongside this technique, enabling researchers to unlock a wealth of information about cellular diversity, gene expression dynamics, developmental processes, and disease mechanisms. By applying single-cell transcriptomics to pancreatic islets and tissue in mice, this thesis aims to to contribute to a deeper understanding of pancreatic islet inflammation and $\beta$-cell dysfunction in the context of T2D





\end{abstract}
