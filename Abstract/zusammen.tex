% ************************** Thesis Abstract *****************************
\begin{zusammen}

\par Typ-2-Diabetes (\glsentryshort{t2d}), eine Stoffwechselstörung, ist mit systemischen Inflammationen verbunden, die auch das Pankreasgewebe und die Langerhans-Inselzellen betreffen. Darüber hinaus ist die kritischste Determinante von T2D die fortschreitende Dysfunktion der Insulin sezernierenden $\beta$-Zellen angesichts der peripheren Insulinresistenz und der gestörten Glukosetoleranz. Unser Verständnis der Dynamik der Infiltration von Insel-Immunzellen während metabolischer Belastungen wie Überernährung und Alterung ist jedoch unvollständig, da keine gezielte Analyse der Immunzellsignaturen durchgeführt wurde und informative Zeitverlaufsdaten fehlen. Darüber hinaus gibt es noch erhebliche Lücken in unserem Verständnis darüber, wie $\beta$-Zellen auf chronischen Stoffwechselstress reagieren und von adaptiven Reaktionen zu $\beta$-Zell-Dysfunktion übergehen. Des WeiterenDarüber hinaus besteht kein Konsens über die Anzahl der $\beta$-Zell-Zustände während Homöostase und Krankheit, was die Notwendigkeit unterstreicht, die Plastizität, die Anpassung und das Versagen von $\beta$-Zellen genauer zu untersuchen.\\

\par In dieser Dissertation verbinde ich das Gebiet der Biologie der Pankreas-Inseln mit Einzelzellanalysen. Unter Verwendung von zuvor etablierten Mausmodellen, welche die menschliche \glsentryshort{t2d}-Pathogenese modellieren, beschreiben wir die Heterogenität und die dynamischen Reaktionen von Immunzellen und $\beta$-Zellen als Reaktion auf verschiedene metabolische Stresssituationen.\\

\par Im ersten Teil meiner Arbeit verwenden wir einen multimodalen Ansatz, der bildgebende Massenzytometrie (\glsentryshort{imc}) und Einzelzell-RNA-Sequenzierung (\glsentryshort{scr}) kombiniert, um die Auswirkungen von Überernährung und Alterung auf die Immunzelldynamik im Pankreasgewebe und den Langerhans-Inseln zu untersuchen. Wir kartieren die räumliche Verteilung und die Transkriptionsprofile von Immunzellen und zeigen signifikante Veränderungen in der Immunlandschaft, einschließlich der beschleunigten Expansion pro-inflammatorischer Populationen und einer stark veränderten Entzündungsreaktion während der Überernährung im Vergleich zum Alterungsprozess.\\

\par Im zweiten Teil meiner Arbeit konstruieren wir einen \glsentryshort{scr}-Atlas von $\beta$-Zellen in verschiedenen Mausmodellen, um die Anpassung und Dysfunktion von $\beta$-Zellen unter erhöhter Arbeitsbelastung und Hyperglykämie zu verstehen. Wir integrieren intern generierte und zuvor veröffentlichte Datensätze in dieser Meta-Analyse und identifizieren verschiedene $\beta$-Zell-Untergruppen. Wir zeigen, dass diese $\beta$-Zellen Untergruppen entlang eines Kontinuums existieren, das der $\beta$-Zell-Dysfunktion entspricht, und dass die $\beta$-Zell-Dysfunktion mit einer erhöhten Arbeitsbelastung und dem Verlust der Zellreifung verbunden ist.\\

\par Insgesamt eröffnet diese Dissertation wertvolle Einblicke in die Komplexität der Entzündungsreaktionen sowie die Anpassung und das Versagen der $\beta$-Zellen bei \gls{t2d} und schafft eine solide Grundlage für weitere Studien zur Milderung der Auswirkungen von Stoffwechselerkrankungen und Verbesserung der klinischen Ergebnisse. 

\end{zusammen}
