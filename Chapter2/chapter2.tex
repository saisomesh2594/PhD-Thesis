%!TEX root = ../thesis.tex
%*******************************************************************************
%****************************** First Chapter *********************************
%*******************************************************************************

\chapter{Thesis Outline}
\label{chp:outline}
%\clearpage
%\newpage\null\thispagestyle{empty}\newpage

Broadly speaking, this thesis weaves together the field of mouse pancreatic islet biology with single-cell transcriptomics. The topic of pancreatic islets is a wide-ranging and highly researched area of biology, with insulin-producing $\beta$-cells at the core, which are responsible for maintaining blood glucose homeostasis in the body. The dysfunction of $\beta$-cells and the inflammation of the pancreatic islets and tissue in the face of peripheral insulin resistance and impaired glucose tolerance eventually lead to the clinical onset of Type-2 diabetes (T2D). Single-cell transcriptomics, a transformative technology, has revolutionized biological research. By analyzing the transcriptome of individual cells, researchers can now routinely dissect the complexities of biological systems at an unprecedented cellular level. Powerful computational tools have emerged alongside this technique, enabling researchers to unlock a wealth of information about cellular diversity, gene expression dynamics, developmental processes, and disease mechanisms. By applying single-cell transcriptomics to pancreatic islets and tissue in mice, this thesis aims to to contribute to a deeper understanding of pancreatic islet inflammation and $\beta$-cell dysfunction in the context of T2D.\\

\par \textbf{\autoref{chp:introduction}} lays groundwork for subsequent research by introducing key concepts. I provide a brief introduction to T2D and its etiology. Then, I introduce the morphology and physiology of the pancreatic tissue and provide a detailed overview of $\beta$-cells from their development and maturation to their crucial role in insulin secretion. I also explore the intriguing idea $\beta$-cell heterogeneity and their adaptive responses to increased workloads. In the next section, I discuss the role of chronic inflammation in pancreas and islets, triggered by obesity and aging, and its contribution to T2D development. Next, I take a deep-dive into single-cell RNA sequencing, its experimental workflow and the computational analysis that follows data generation. I also review some of the recent single-cell studies focusing on the pancreas. Finally, I conclude this chapter with a concise overview of imaging mass cytometry, its workflow and data analysis steps.\\
\par \textbf{\autoref{chp:diet_aging}} focuses on detailed analysis of immune cell dynamics within pancreatic tissue and pancreatic islets.\\ 
\par \textbf{\autoref{chp:meta_analysis}}




% $\beta$-cells are the only 


% to investigate islet inflammation in response to chronic over-nutrition and aging, and to understand how islet $\beta$-cells initiate compensatory responses in several models of increasing workload and how $\beta$-cells fail at the onset of decompensation. 

% investigation of pancreatic islet inflammation and $\beta$-cell dysfunction, on a single-cell level, in the context of \gls{t2d} pathogenesis, using mouse models.  microenvironment in response to 
