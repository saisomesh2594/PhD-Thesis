%!TEX root = ../thesis.tex
%*******************************************************************************
%****************************** First Chapter *********************************
%*******************************************************************************

\chapter{Thesis Outline}
\label{chapter1}
\newpage

Broadly speaking, this thesis weaves together the field of mouse pancreatic islet biology with single-cell RNA sequencing. The topic of pancreatic islets is a wide-ranging and highly researched area of biology, with \textit{insulin}-producing $\beta$-cells at the core, which are responsible for maintaining blood glucose homeostasis in the body. The dysfunction of $\beta$-cells combined with insulin resistance in the body and inflammation of the pancreatic islets and tissue eventually lead to the clinical onset of Type-2 diabetes. Single-cell transcriptomics is a well-established technology and has revolutionized research in biology. This method is now routinely applied to investigate several mechanisms of complex systems on a cellular level. Alongside this, computational methods for the analysis of single-cell data have developed at breakneck speeds and have enabled researchers to obtain detailed insights from the datasets.\\\\



$\beta$-cells are the only 


to investigate islet inflammation in response to chronic over-nutrition and aging, and to understand how islet $\beta$-cells initiate compensatory responses in several models of increasing workload and how $\beta$-cells fail at the onset of decompensation. 

investigation of pancreatic islet inflammation and $\beta$-cell dysfunction, on a single-cell level, in the context of \gls{t2d} pathogenesis, using mouse models.  microenvironment in response to 
