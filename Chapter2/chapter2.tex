%!TEX root = ../thesis.tex
%*******************************************************************************
%****************************** First Chapter *********************************
%*******************************************************************************

\chapter{Thesis Outline}
\label{chp:outline}
%\clearpage
%\newpage\null\thispagestyle{empty}\newpage

%Broadly speaking, this thesis weaves together the field of mouse pancreatic islet biology with single-cell transcriptomics. The topic of pancreatic islets is a wide-ranging and highly researched area of biology, with insulin-producing $\beta$-cells at the core, which are responsible for maintaining blood glucose homeostasis in the body. The dysfunction of $\beta$-cells and the inflammation of the pancreatic islets and tissue in the face of peripheral insulin resistance and impaired glucose tolerance eventually lead to the clinical onset of Type-2 diabetes (T2D). Single-cell transcriptomics, a transformative technology, has revolutionized biological research. By analyzing the transcriptome of individual cells, researchers can now routinely dissect the complexities of biological systems at an unprecedented cellular level. Powerful computational tools have emerged alongside this technique, enabling researchers to unlock a wealth of information about cellular diversity, gene expression dynamics, developmental processes, and disease mechanisms. By applying single-cell transcriptomics to pancreatic islets and tissue in mice, this thesis aims to to contribute to a deeper understanding of pancreatic islet inflammation and $\beta$-cell dysfunction in the context of T2D.\\

% \vspace{-20pt}
\par This thesis is a monograph and is structured into six main chapters and three appendix chapters.\\

\par \textbf{\autoref{chp:introduction}} lays the groundwork by providing detailed biological background and a broad overview about single-cell technologies employed in this thesis. \textbf{\autoref{chp:aims}} presents the aims of this thesis work.\\

\par \textbf{\autoref{chp:diet_aging}} and \textbf{\autoref{chp:meta_analysis}} form the core of this thesis, presenting the main results and findings. In \textbf{\autoref{chp:diet_aging}}, we investigate the impact of overnutrition and aging on pancreatic immune cells using a multi\textit{-omics} approach. In \textbf{\autoref{chp:meta_analysis}}, we perform a comprehensive meta-analysis of pancreatic $\beta$-cells across various models of $\beta$-cell workload and hyperglycemia.\\ %Together, these chapters offer key contributions to understanding the pathogenesis of metabolic disorders, such as \glsentrylong{t2d}.\\


% \par In \textbf{\autoref{chp:introduction}}, I lay the groundwork by introducing several key topics. I provide a brief introduction to \gls{t2d} and its etiology. Then, I introduce the morphology and physiology of the pancreatic tissue and provide a detailed overview of $\beta$-cells, from their development and maturation to their crucial role in insulin secretion. I also explore the intriguing topic of $\beta$-cell heterogeneity and the adaptive responses by $\beta$-cells in challenging conditions. In the next section, I discuss the role of chronic inflammation in pancreas and islets triggered by obesity and aging, and its contribution to the development of \gls{t2d}. Next, I take a deep-dive into \gls{scr} and discuss in length the various computational methods that enable downstream data analysis. I also provide a brief overview of few of the single-cell methods that explore other modalities and also review some of the recent single-cell studies fon the pancreas. Finally, I conclude this chapter with a concise overview of \acrfull{imc}, its associated workflow and the data analysis steps.\\

% \par In \textbf{\autoref{chp:diet_aging}}, we investigated the impact of \acrfull{wd} feeding and aging on pancreatic immune cells using a multi\textit{-omics} approach, combining \gls{scr} and \gls{imc}. The integration of \gls{scr} and \gls{imc} data provided a comprehensive spatio-temporal profile of immune cell dynamics, revealing diet-specific and age-specific inflammatory patterns and their roles in the development of \gls{t2d}. We were able to depict that \gls{wd} feeding accelerated the age-dependent accumulation of inflammatory macrophage and T-cell sub-populations and modulated an aging-specific \gls{ifn} response. The findings from this chapter emphasized the necessity of considering both dietary and age-related factors in diabetes research to understand their combined effects on pancreatic inflammation and disease progression.\\ 

% \par In \textbf{\autoref{chp:meta_analysis}}, we conducted a comprehensive meta-analysis of seven \gls{scr} islet studies to create an integrated transcriptomic atlas of $\beta$-cells. By including both in-house and publicly available datasets, we identified and characterized distinct $\beta$-cell subsets across various models of increased $\beta$-cell workload and hyperglycemia. This analysis revealed consistent transcriptional responses across the various models, highlighting how chronic metabolic stress influences $\beta$-cell functionality and contributes to $\beta$-cell dysfunction. Additionally, we explored \glspl{grn} underlying these transcriptional changes, highlighting known and identifying novel \glspl{tf} involved in $\beta$-cell adaptation and dysfunction. Our findings demonstrated the utility of this integrated atlas for mapping additional datasets and understanding $\beta$-cell heterogeneity, providing valuable insights into the molecular mechanisms of \gls{t2d}.\\

\par \textbf{\autoref{chp:discussion}} summarizes the observations and findings from the previous two chapters. Following this, I place and discuss these findings in a broader scientific context, highlighting their implications and relevance to the field. I also outline the limitations of this work and suggest future research directions.\\

\par \textbf{\autoref{chp:methods}} encompasses an overview of the computational steps for analyzing the single-cell data generated in this work.\\

\par \textbf{\autoref{chp:app_tables}} and \textbf{\autoref{chp:app_figures}} include additional tables and figures to complement the main results, and \textbf{\autoref{chp:app_packages}} lists all the packages used for analyzing the data. All references cited in this thesis are combined into a common \hyperref[bibliography]{\textbf{Bibliography}}. 




% $\beta$-cells are the only 


% to investigate islet inflammation in response to chronic over-nutrition and aging, and to understand how islet $\beta$-cells initiate compensatory responses in several models of increasing workload and how $\beta$-cells fail at the onset of decompensation. 

% investigation of pancreatic islet inflammation and $\beta$-cell dysfunction, on a single-cell level, in the context of \gls{t2d} pathogenesis, using mouse models.  microenvironment in response to 
