\chapter{Aims}
\label{chp:aims}

\newpage\null\thispagestyle{empty}\newpage

\par The work presented in this thesis is focused on investigating two crucial topics in the field of \acrfull{t2d} pathogenesis:\\

\textbf{\underline{Obesity, aging and pancreatic inflammation}}
\vspace{15pt}
\par Chronic nutrient excess leads to progressive dysfunction of insulin-secreting $\beta$-cells, resulting in glucose intolerance and eventually \glsentryshort{t2d}. Another risk factor for the development of \glsentryshort{t2d} is aging, and both are associated with a chronic, low-grade inflammation in metabolic tissues, including pancreatic islets. Islet inflammation is a hallmark of \glsentryshort{t2d} development and pathology. However, a detailed survey of how nutrient excess and aging impacts immune cell dynamics within the islets is lacking.\\

\par The goal of this project was to conduct in-depth and integrative data analysis from multi-parametric technologies such as \glsentrylong{scr} (\glsentryshort{scr}) and imaging mass cytometry (\glsentryshort{imc}) to assess the impact of obesity (induced by a a\glsentrylong{wd}) and aging on CD45\textsuperscript{+} immune cells in pancreas and pancreatic islets. The multi-modal approach provided a spatio-temporal profile of immune cell dynamics, revealing diet-specific and age-specific inflammatory patterns and their roles in the development of \glsentryshort{t2d}. The findings from this chapter emphasized the necessity of considering both dietary and age-related factors in diabetes research to understand their combined effects on pancreatic inflammation and disease progression.\\

\textbf{\underline{$\beta$-cell sub-populations and metabolic stress responses}}
\vspace{15pt}
\par Heterogeneity in $\beta$-cells is crucial for adaptive and compensatory responses in face of metabolic challenges. While \glsentryshort{scr} studies have detailed this heterogeneity, there is a general lack of consensus on the number of $\beta$-cell sub-populations and their interrelationships. Furthermore, a thorough understanding of how $\beta$-cells initiate adaptive responses and how they fail due to unresolved stress is lacking.\\

\par The goal of this project was to perform a comprehensive analysis of the heterogeneous $\beta$-cell subsets across various models of $\beta$-cell adaptation and decompensation. We created an integrated transcriptomic atlas of $\beta$-cells and identified distinct $\beta$-cell subsets across various models of increased $\beta$-cell workload and hyperglycemia. We further characterize the relationship between these $\beta$-cell subsets as they respond to increasing metabolic demands. Overall, our integrated atlas serves as a valuable resource for the analysis of maladaptive $\beta$-cell responses to various \glsentryshort{t2d} stressors, and underscores the complex and dynamic nature of $\beta$-cells under \gls{t2d} conditions.

% \par In \textbf{\autoref{chp:meta_analysis}}, we conducted a comprehensive meta-analysis of seven \gls{scr} islet studies to create an integrated transcriptomic atlas of $\beta$-cells. By including both in-house and publicly available datasets, we identified and characterized distinct $\beta$-cell subsets across various models of increased $\beta$-cell workload and hyperglycemia. This analysis revealed consistent transcriptional responses across the various models, highlighting how chronic metabolic stress influences $\beta$-cell functionality and contributes to $\beta$-cell dysfunction. Additionally, we explored \glspl{grn} underlying these transcriptional changes, highlighting known and identifying novel \glspl{tf} involved in $\beta$-cell adaptation and dysfunction. Our findings demonstrated the utility of this integrated atlas for mapping additional datasets and understanding $\beta$-cell heterogeneity, providing valuable insights into the molecular mechanisms of \gls{t2d}.\\