%!TEX root = ../thesis.tex
%*******************************************************************************
%****************************** Sixth Chapter *********************************
%*******************************************************************************

\chapter{Discussion}  %Title of the Sixth Chapter
\label{chapter6}
\clearpage

\newpage\null\thispagestyle{empty}\newpage

% \par \acrfull{t2d} is a chronic metabolic disorder characterized by peripheral insulin resistance and a relative insulin deficiency. A critical facet of \gls{t2d} is the progressive dysfunction of pancreatic islet $\beta$-cells that secrete insulin which plays an indispensable role in the regulation of glucose homeostasis. The onset of $\beta$-cell dysfunction in \gls{t2d} may involve different but synergistic mechanisms which may trigger inflammation. The chronic, low-grade inflammation of metabolic tissues, including pancreas and pancreatic islets, is now considered to be a part of the etiology of \gls{t2d}. Cross-talk between immune cells and $\beta$-cells in turn can initiate a vicious cycle of $\beta$-cell dysfunction. However, our understanding of the dynamics of islet immune cell infiltration during \gls{t2d} progression is incomplete as no focused analysis of immune cell signatures has been performed and time-course data in informative animal models are missing. While the role of islet-resident macrophages in maintaining islet homeostasis and driving islet inflammation has been appreciated for quite a while, knowledge about their phenotypes and their dual role \\

\par \acrfull{t2d} is a complex disorder accounting for over 90\% of all diabetic cases worldwide. In \gls{t2d}, increased blood glucose levels, or hyperglycemia, is the result of body's inability to respond fully to insulin, a condition termed as insulin resistance. The onset of insulin resistance increases insulin demand leading to increased insulin synthesis and secretion by $\beta$-cells which are located in cell clusters known as islets, within the pancreas. Chronic insulin demand eventually results in $\beta$-cell dysfunction and failure, leading to overt \gls{t2d}.\\
\par Obesity and aging are key risk factors for \gls{t2d}, and both conditions are associated with persistent, low-level inflammation in metabolic tissues including pancreas and pancreatic islets. This type of inflammation is often referred to as metaflammation in obesity and inflammaging in aging, and are now considered to be a part of the etiology of \gls{t2d}. Crosstalk between immune cells and $\beta$-cells under metabolic stress can initiate vicious cycles of $\beta$-cell damage. While accumulating evidence has implicated the innate immune system, particularly islet-resident macrophages in driving the metabolic-stress induced inflammation, discrepancies about their nature, number and inflammatory markers in islets of \gls{t2d} patients and rodent models still exist. Furthermore, the association of adaptive immune system, particularly T-cells, to \gls{t2d} pathogenesis is highly debated with studies providing conflicting findings, thereby warranting additional investigation. Thus, our understanding of the dynamics of islet immune cell infiltration during \gls{t2d} progression is still incomplete as no focused analysis of immune cell signatures has been performed and informative time-course data is missing.\\
\par In the face of insulin resistance, $\beta$-cells increase their insulin secretion to maintain normal blood glucose levels, also known as $\beta$-cell compensation. The compensatory mechanisms, although not completely elucidated, involve $\beta$-cell mass expansion, increased insulin synthesis and secretion, enhanced glucose sensing and augmented antioxidative functions in $\beta$-cells, all of which act alone or together to overcome insulin resistance and maintain normoglycemia. However, this ability to compensate is transient and the production of large amounts of insulin by compensating $\beta$-cells exerts enormous pressure on the protein production machinery within the $\beta$-cells. Ultimately, $\beta$-cells are unable to keep up with the increased workload and the initial adaptive responses give way to maladaptive processes. 
 The critical determinant for \gls{t2d} is $\beta$-cell dysfunction, which has been extensively studied in insulin-resistant conditions and in \gls{t2d}, both in humans and rodent models. However, a comprehensive understanding of how $\beta$-cells initiate compensatory responses and how they fail at the onset of decompensation as a consequence of unresolved stress is still unknown.\\
 \par The field of single-cell biology, powered by advanced single-cell \textit{-omics} technologies, and combined with next-generation sequencing methods has revolutionized our understanding of cells. These technologies allow us to examine cellular heterogeneity, revealing unexpected diversity within groups previously thought to be uniform. This has shed light on how specific cell-types contribute to health, disease development, and tissue function. These single-cell \textit{-omics} methods are now widely used in research thereby enabling scientists to routinely investigate cellular and molecular profiles of tissues and further enhance our biological understanding.\\
 
%As a way to shield themselves from stress, considerable evidence has demonstrated the loss of functional mature identity in stressed rodent and human $\beta$-cells, termed as dedifferentiation as well as the switch to other endocrine hormone expressing cell-types, known as transdifferentiation. This points to the remarkable plasticity of $\beta$-cells that allows them to variably respond depending on the environmental 

%\clearpage

% Enhanced $\beta$-cell function via increased glucose metabolism The critical determinant for \gls{t2d} is $\beta$-cell dysfunction which is accompanied by loss of mature functional identity % The mechanisms associated with progressive $\beta$-cell dysfunction may trigger inflammation and crosstalk between immune cells and $\beta$-cells can initiate vicious cycles. of $\beta$-cell damage. The persistent, low-level inflammation of metabolic tissues, including pancreas and pancreatic islets, is now considered to be a part of the etiology of \gls{t2d}.% be initiated The mechanisms that initiate and failure in \gls{t2d} , resulting from chronic periods of increased insulin demand.  \\

\par The work presented in this thesis lies at the intersection of two pivotal topics: 
\begin{itemize}
    \item mouse pancreatic tissue and pancreatic islet inflammation in response to metabolic stresses such as obesity and aging \textit{and}
    \item the heterogeneity and compensatory responses of mouse islet $\beta$-cells to increased workloads and hyperglycemia\\
\end{itemize}

Using high-throughput single-cell technologies, we have conducted a through investigation into the two critical etiologies of \gls{t2d}: inflammation and $\beta$-cell dysfunction. The findings from this thesis work pave the way for a better understanding of the underlying mechanisms that drive \gls{t2d} and open new avenues for targeted therapeutic strategies. This comprehensive work also contributes significantly to the evolving field of diabetes research, highlighting the potential of single-cell technologies to dissect complex biological processes and potentially inform clinical interventions.

\section{Summary of findings}

\subsection{\autoref{chp:diet_aging}}

\par In \textbf{\autoref{chp:diet_aging}}, we utilized a multi\textit{-omics} approach, integrating imaging mass cytometry (\glslink{imc}{IMC}) and \acrfull{scr} to investigate how metabolic stresses such as \acrfull{wd}-induced obesity and aging influence immune cell dynamics in mice pancreatic tissue and islets. For \gls{imc}, we established a panel of markers \textbf{(\autoref{tab:app_imc_panel})} that enabled the identification of diverse immune cell-types and sub-populations within the pancreatic tissue. The \gls{imc} data analysis utilized sophisticated computational pipeline and image segmentation techniques which facilitated the mapping of spatial distributions of immune cell-types in the pancreatic tissue at single-cell resolution. For \gls{scr}, We performed CD45 (\textit{Ptprc}) enrichment in order to recover considerable number of immune cells from isolated pancreatic islets while also recovering the endocrine cells from the islets \textbf{(\autoref{fig:chp2_experimental_design})}. Subsequently, these cells were sequenced and following rigorous computational analysis, a high-resolution transcriptional profile of these cells was achieved. The integration of these two modalities resulted in a comprehensive atlas of immune cells within pancreatic tissue and islets during metabolic stress induced by overnutrition and aging \textbf{(\autoref{fig:chp2_fullscRNA} B)}. Furthermore, with the single-cell data, we also extensively characterized the CD45\textsuperscript{-} islet cells which consisted of endocrine and exocrine populations, endothelial cells, and other rare cell-types within pancreatic islets \textbf{(\autoref{fig:chp2_fullscRNA} D)}. 

% We applied this approach to 
% establish a comprehensive atlas of immune cells within pancreatic tissue and islets during metabolic stress induced by \acrfull{wd} and aging. %This resource integrates the phenotypic characteristics of these immune cells with their spatial localization, thereby providing a more comprehensive understanding of their roles under these conditions. 
% In line with existing studies, we performed a detailed characterization of macrophage sub-populations in the pancreatic tissue and within the pancreatic islets. Additionally, the diverse immune panel and the CD45\textsuperscript{+} \glslink{facs}{FACS} enrichment enabled the identification and detailed annotation of the previously uncharacterized T-cell sub-populations in the pancreas, thereby extending our knowledge of this immune cell repertoire in this niche.\\
% \par We discovered that \gls{wd}-induced obesity significantly accelerated the age-dependent accumulation of the 
% %multiple immune cell populations within the pancreas and around the pancreatic islets. These populations included the 
% inflammatory F4/80\textsuperscript{\textit{low}} macrophages and the CD8\textsuperscript{+} cytotoxic T-cells. The accelerated inflammation during overnutrition is evident in rising numbers of these sup-populations as well as in their altered molecular profiles and enhanced intercellular communication, all of which likely perpetuate local islet inflammation. Interestingly, the \gls{wd}-accelerated inflammation deviated from typical age-associated inflammatory patterns and exhibited an unique type-1 \glslink{ifn}{IFN} response. These findings further underscore the complex interplay between overnutrition and inflammation that contributes to the complex pathogenesis of \gls{t2d}.
% % \par In addition to the multi-modal characterization of immune cells, \gls{scr} analysis of $\beta$-cells revealed that both, metabolic stress and aging, enhanced \gls{oxphos} machinery, but at the expense of down-regulation of $\beta$-cell identity markers.

\subsubsection{Diet-induced obesity accelerates the aging-dependent accumulation of inflammatory macrophages}

\par We comprehensively profiled the islet macrophage sub-populations, examining their activation status and dynamics in the context of \gls{wd} feeding \textbf{(\autoref{fig:chp2_scrna_macrophages} A)}. Our findings revealed two distinct pro-inflammatory macrophage sub-populations characterized by low F4/80 expression and each with unique activation statuses: the F4/80\textsuperscript{\textit{low}} Macs-3 which shows significant type-1 \glsreset{ifn} \gls{ifn} activation, and the F4/80\textsuperscript{-} Macs-2 identified as the primary source of the type-1 \gls{ifn} \textbf{(\autoref{fig:chp2_scrna_macrophages} B,C)}. While the proportion of Macs-2 remained relatively unchanged during \gls{wd} feeding and aging, importantly, Macs-3 was the only macrophage sub-population that depicted significant expansion in the islets in response to overnutrition, and a minor increase during aging \textbf{(\autoref{fig:chp2_scrna_macrophages} D)}. This aligned with the trend depicted by the corresponding F4/80\textsuperscript{\textit{low}} macrophages in the \gls{imc} analysis which accumulated in the pancreatic tissue as well as in the islets and the peri-islet region \textbf{(\autoref{fig:chp2_imc_macrophages} C,} middle\textbf{; D,} middle\textbf{)}. Besides these two activated sub-populations, we also identified a homeostatic Macs-1, a proliferative Macs-4 and a phagocytotic Macs-5 sub-populations \textbf{(\autoref{fig:chp2_scrna_macrophages})}. This heterogeneity in islet macrophages and their activation status which resembled to a previous finding in a \gls{t1d} \gls{nod} mouse model \textbf{(\autoref{fig:chp2_scrna_macrophages_unanue})} \textbf{\cite{zakharov_single-cell_2020}}, had not been fully explored in the context of obesity or \gls{t2d} until now By comparing the expression of genes corresponding to the markers in \gls{imc} channels, we were able to link the islet-resident macrophages from our \gls{scr} data to the macrophages identified with the \gls{imc} analysis \textbf{(\autoref{fig:chp2_scrna_macrophages_imc})}.\\   % Future studies exploring activation differences between \gls{t1d} and \gls{t2d} could elucidate distinct islet inflammation mechanisms in these conditions.\\

\par We further characterized the responses of the two inflammatory macrophage sub-populations to \gls{wd} feeding and aging in our \gls{scr} data. We observed that \gls{wd} feeding resulted in a stronger expression of genes coding for cytokines and chemokines in Macs-2 despite their lower activation status compared to Macs-3, and this response was not evident during aging \textbf{(\autoref{fig:app_scrna_macrophages_macs2_dge})}. We also discovered that both aging and overnutrition conditions resulted in the activation of the type-1 \gls{ifn} response in Macs-3 \textbf{(\autoref{fig:chp2_scrna_macrophages_macs3_dge})}. Intriguingly, our analysis revealed distinct type-1 \gls{ifn} responses in Macs-3 dependent on the stimulus:  while aging induced a canonical response with \glslink{stat1}{STAT1} activation, overnutrition led to a non-canonical response marked by \glslink{stat3}{STAT3} activation \textbf{(\autoref{fig:chp2_scrna_macrophages_macs3_dge})} and the involvement of alternative pathways such as \gls{mtor} signaling \textbf{(\autoref{fig:chp2_scrna_macrophages_macs3_clust} B)}, thereby likely leading to more pronounced inflammatory state. Additionally, the type-1 \gls{ifn} responsiveness in overnutrition was linked to the expression of \textit{Cxcl10}, T-cell attracting chemokines and inflammatory cytokines \textbf{(\autoref{fig:chp2_scrna_macrophages_macs3_dge} A,B; \autoref{fig:app_scrna_macrophages_macs2_dge} A,B)}.
% Given that \glslink{stat3}{STAT3} is known to modulate the intensity and shift the type-1 \gls{ifn} response from canonical to non-canonical \textbf{\cite{tsai_fine-tuning_2019}}, our findings suggest that the type-1 \gls{ifn} response in islet-associated macrophages during metaflammation may be more strongly altered compared to the response in inflammaging. Whereas in aging, macrophages only transcribed the type-1 \gls{ifn}-induced chemokine \textit{Cxcl9}, type-1 \gls{ifn} responsiveness in overnutrition was linked to the expression of \textit{Cxcl10}, T- cell attracting chemokines and inflammatory cytokines \textbf{(\autoref{fig:chp2_scrna_macrophages_macs3_dge} A,B; \autoref{fig:app_scrna_macrophages_macs2_dge} A,B)}

\subsubsection{T-cells}
Utilizing the diverse panel of immune markers in our \gls{imc} analysis and the CD45 enrichment approach for \gls{scr}, we identified a wide variety of CD3 (\textit{Cd3e}) expressing T-cell sub-populations within the pancreatic tissue and the islets \textbf{(\autoref{fig:chp2_imc_umap}; \autoref{fig:chp2_scrna_tcells1} A,C)}. Notably, CD8\textsuperscript{+} activated effector-like T-cells exhibited a pronounced expansion under overnutrition and these cells were also significantly enriched in the peripheral regions of the islet under \gls{wd} feeding, but not during aging \textbf{(\autoref{fig:chp2_imc_tcells1})}. The CD8\textsuperscript{+} cytotoxic T-cells in the \gls{scr} data which depicted a strong alignment with the CD8\textsuperscript{+} activated effector-like T-cells from the \gls{imc} analysis \textbf{(\autoref{fig:app_scrna_tcells1} A,B)} also expanded in response to \gls{wd} feeding \textbf{(\autoref{fig:chp2_scrna_tcells1} B)}. However, contrary to the observations from \gls{imc}, the CD8\textsuperscript{+} cytotoxic T-cells expanded significantly during aging \textbf{(\autoref{fig:chp2_scrna_tcells1} B)}.

\subsubsection{Perpetuating local islet inflammation}
We next surveyed the complex intercellular communication landscape between macroph-ages and T-cells in our \gls{scr} data during the course of \gls{wd} feeding. We observed that an acute \gls{wd} feeding regimen resulted in an increase the number and strength of signaling into the two activated macrophage sub-populations (Macs-2 and Macs-3) compared to their chow diet controls \textbf{(\autoref{fig:chp2_scrna_cellchat1} A; \autoref{fig:app_scrna_cellchat1} A)}. We further examined the \glsreset{lri} \gls{wd} up-regulated \glspl{lri} between Macs-2 and Macs-3 and identified enhanced signaling between the \textit{Ifnb1} expressing Macs-2 to the \gls{ifn}-responsive Macs-3, which expressed \gls{ifn}A receptors \textbf{(\autoref{fig:chp2_scrna_cellchat1} B)}. Compared to 12 weeks of chow diet feeding, Macs-3 depicted increased  number of interactions in response to 12 weeks of \gls{wd} feeding, and particularly to the CD8\textsuperscript{+} cytotoxic T-cells \textbf{(\autoref{fig:chp2_scrna_cellchat2} A)}. 
%However, the strength of these incoming and outgoing interaction in Macs-3 were much pronounced compared to short term overnutrition \textbf{(\autoref{fig:app_scrna_cellchat1} B,} middle\textbf{)}. 
Notably, the possible signalling from Macs-3 to CD8\textsuperscript{+} cytotoxic T-cells involved the action of \textit{Ccl5} secreted by Macs-3 onto the \textit{Ccr5} receptor on the CD8\textsuperscript{+} cytotoxic T-cells, with the assistance of \textit{Cd86-Cd28} co-stimulatory interactions \textbf{(\autoref{fig:chp2_scrna_cellchat2} B,C)}. Based on the \gls{dge} analysis of Macs-2 sub-population across \gls{wd} feeding and aging \textbf{(\autoref{fig:app_scrna_tcells1})}, Macs-2 also likely depicted a non-canonical type-1 \gls{ifn} response to \gls{wd} feeding and thereby contributing to the macrophage - T-cell crosstalk within the pancreatic islet niche. This was further supported by the positive correlation between the number of F4/80\textsuperscript{-} Macs-2 and F4/80\textsuperscript{\textit{low}} Macs-3 macrophages with the CD8\textsuperscript{+} activated effector-like T-cells across all \glspl{roi} in the \gls{imc} data \textbf{(\autoref{fig:chp2_imc_correlation})}.\\
\par Taken together, our findings revealed that both, \gls{wd} and natural aging distinctly influenced the immune landscape, contributing to significant shifts in composition, function and the spatial organization of immune cells. Furthermore, our analysis indicated active communication between components of the innate and adaptive immune system via distinct profiles under overnutrition and aging conditions. These findings enhance our understanding of the cellular and molecular mechanisms underlying metabolic stress induced inflammation during \gls{t2d} progression.



%Based on the observations of \gls{wd}-induced accumulation of the \gls{ifn} responsive inflammatory Macs-3 macrophages and the CD8\textsuperscript{+} activated effector-like T-cells, we surveyed the complex intercellular communication between these sub-populations during the course of \gls{wd} feeding. 
% \clearpage

% \subsection{\autoref{chp:meta_analysis}}

% \par In \textbf{\autoref{chp:meta_analysis}}, we compiled a high-quality, integrated, single-cell transcriptomic atlas of $\beta$-cells across several mouse models of $\beta$-cell adaptation and decompensation. The overall aim of this meta-analysis study was in understanding the mechanisms that underlie the compensatory responses by $\beta$-cells in response to increased workloads and how $\beta$-cell fail at the onset of decompensation. We included five in-house \gls{scr} datasets and complemented these with two previously published datasets, thus incorporating models with broad ranges of $\beta$-cell workload and hyperglycemia.\\

% \subsubsection{Mouse models of $\beta$-cell decompensation closely mimic human \gls{t2d} and elicit similar transcriptional responses to hyperglycemia}

% \par The exploration of this curated $\beta$-cell transcriptome resource revealed that transcriptional signatures activated by increased workload and repressed by hyperglycemia are consistent across models. Additionally, we identified model-specific signatures that underscore distinct molecular adaptations and pathophysiological responses of $\beta$-cells to metabolic stress thereby offering insights into $\beta$-cell dysfunction and failure. Further, we assessed how closely these models mimicked the transcriptional signatures of human \gls{t2d} by examining overall expression trends of gene-sets enriched in human \gls{t2d}. We identified that $\beta$-cell compensatory responses resulted in the down-regulation of genes related to $\beta$-cell identity and function across all models. Similar to human \gls{t2d} pathogenesis, mouse models of increased $\beta$-cell workload and dysfunction, in the context of mild and severe hyperglycemia, induced pathways related to \glslink{er}{ER} and metabolic stress and up-regulated cellular machinery associated with ribosomal biogenesis and vesicle trafficking.

% \subsubsection{Islet $\beta$-cells are transcriptionally heterogeneous}
% \par Next, we used this integrated atlas to describe the heterogeneity of the $\beta$-cell transcriptomic landscape and relate the $\beta$-cell subtypes, identified in the individual analyses of the models, in an unified framework. We identified $\beta$-1 Normal subtype expressing known markers of maturity and function and enriched in the unchallenged controls of all models and the 2 year old mice. Additionally, this subtype was characterized by higher activity of the \textit{Neurod1} and the \textit{Nfia} regulons. Conversely, the $\beta$-3 Stress-immature subtype expressed markers related to cellular stress and exhibited stress-associated dedifferentition evidenced by the expression of known $\beta$-cell immaturity markers. This enrichment of this subset was evident in severe hyperglycemic models such as the \textit{db/db} and \textit{ob/ob} mice as well as in response to insulin receptor blockade by S961 and partial $\beta$-cell ablation by \glslink{stz}{STZ}. The characterization for this subset was further supported by the higher regulon activity of \textit{Arx} and \textit{Rorc} and down-regulation of \textit{Neurod1, Vdr}\\
% Most importantly, our analysis revealed a subset of $\beta$-cells present in all samples and exhibited universal enrichment across all models with increasing workload. The $\beta$-2 Compensating subset showed an up-regulation of the \glslink{er}{ER} machinery which likely points to the adaptive response by $\beta$-cells in order to maintain homeostasis. 

\section{Discussion of findings}

\subsection[Metaflammation and Inflammaging: two sides of the same coin]{Metaflammation and Inflammaging:\\ two sides of the same coin}

\vspace{20pt}

\par In contemporary discussions about progression of chronic disorders, the concepts of metaflammation and inflammaging have garnered considerable attention in understanding how systemic inflammation is interconnected with metabolic and aging processes. Metaflammation which describes metabolic inflammation in response to excess nutrients and obesity is intricately linked to inflammaging which describes a form of chronic, low-grade sterile inflammation associated with chronological aging \textbf{\cite{}}. Interestingly, these two conditions share similar phenotypes: elevated levels of pro-inflammatory markers, macrophages which are the primary mediators of inflammation and metabolic dysfunction resulting in insulin resistance and hyperglycemia \textbf{\cite{https://doi.org/10.1016/j.arr.2017.10.003}}. The similarities between metaflammation and inflammaing have been primarily documented at the systemic level, and has been examined in detail particularly in the adipose tissues, as they are considered to be the main source of pro-inflammatory factors in obesity and aging \textbf{\cite{hotamisligil_inflammation_2017,https://doi.org/10.1016/j.arr.2017.10.003}}. While abundant evidence also demonstrates that inflammatory processes are activated in pancreatic islets, research on inflammation in pancreas during aging is limited.
\clearpage
\par Our study addressed this gap by conducting a comprehensive analysis of inflammaging in the pancreas and pancreatic islets of mice during the course of normal aging process. Furthermore, the inclusion of the \acrfull{wd}-induced obesity axis allowed for the direct comparison of the inflammatory processes elicited in metaflammation and inflammaging, which to our knowledge is a first in this domain. The detailed characterization of the heterogeneity of pancreatic macrophages in the context of metabolic stresses adds to the growing list of studies that describe a continuum of activation of islet macrophages during disease progression. Similar efforts have also been undertaken to describe macrophages and their relationship to low-grade inflammation, insulin resistance and hyperinsulinemia in adipose tissues and liver as well as in pancreatic islets of \gls{t1d} \gls{nod} mice \textbf{\cite{zakharov_single-cell_2020}}. To further reinforce these findings, future research should aim to exhaustively characterize the responses of different sub-populations of macrophages in healthy and metabolically inflamed tissues to physiological and pathological cues.\\

\par Furthermore, our analysis, demonstrated that T-cell accumulation within pancreatic islets is linked to aging while the \gls{wd} model confirms the accumulation in the exocrine pancreas during overnutrition. However, further studies are warranted to confidently ascertain a role for the adaptive immune system in obesity-associated islet pathology. While this remains a subject of ongoing research and debate, a major constrain is the sparse presence of adaptive immune cells such as B-cells and T-cells within the pancreas, that makes the characterization of their abundance, distribution and dynamics challenging. Nevertheless, our integrative analysis also offered detailed annotations and phenotypic characterizations of other rare immune cell populations within the pancreas and pancreatic islets, such as innate lymphoid cells [PMID: 29166590] and dendritic cells [PMID: 18427107, 33912139]. This enriched dataset serves as a valuable resource for elucidating the roles these cells play in the mechanisms of obesity and aging, potentially unlocking new pathways for understanding metabolic diseases and their progression.\\

\par Most importantly, our study demonstrated shared inflammatory pathways between metabolic stress and aging and complements the concept that metaflammation significantly contributes to and accelerates inflammaging \textbf{\cite{franceschi_continuum_2018}}. This occurs via the accelerated accumulation of a pro-inflammatory macrophage sub-population within the islets and the peri-islet region and the activation of a type-1 \acrfull{ifn} response that is similar yet distinct during overnutrition and aging. In the context of diabetes research, much attention has been focused toward the role of type-1 \gls{ifn} signaling in \gls{t1d}, wherein its involvement in driving autoimmune responses is well-documented \textbf{\cite{marroqui_type_2021,newby_type_2017}}. Among the type-1 \glspl{ifn}, the contribution of \gls{ifn}-$\alpha$ to the development and pathogenesis of \gls{t1d} is particularly crucial, leading to overexpression of antigen presentation genes in humans (HLA-I) and mouse (MHC-I) alike, \gls{er} stress and $\beta$-cell apoptosis within pancreatic islets \textbf{\cite{marroqui_interferon-_2017,marro_progression_2017,lombardi_interferon_2018,colli_integrated_2020,jiang_interferon-_2022,coomans_de_brachene_interferons_2024}}. Furthermore, the pro-inflammatory and regulatory roles of type-2 \gls{ifn} signaling via \gls{ifn}-$\gamma$ in \gls{t1d} has also been noted \textbf{\cite{coomans_de_brachene_interferons_2024,de_george_inflammation_2023}}.\\
\par However the involvement of \gls{ifn} signaling in \gls{t2d} pathogenesis is not yet fully recognized. Independent studies have highlighted the role of pro-inflammatory cytokine \gls{ifn}-$\gamma$ in attenuating insulin signaling in adipocytes \textbf{\cite{https://pubmed.ncbi.nlm.nih.gov/19776010/}} inducing adipose tissue inflammation and endothelial dysfunction in \gls{t2d} \textbf{\cite{https://www.doi.org/10.1007/S00395-011-0212-X}} and impairing energy expenditure in skeletal muscles \textbf{\cite{https://academic.oup.com/nar/article/40/4/1609/2411805?login=false#83644092}}. Our study highlights a previously unrecognized role for \gls{ifn}-$\beta$, a type-1 \gls{ifn}, in islet macrophages during \gls{wd}-induced obesity and aging, and provides a novel insight into the pathogenesis of \gls{t2d}. Furthermore, our findings also suggest that the type-1 \gls{ifn} response in islet-associated macrophages during metaflammation may be more strongly altered compared to the response in inflammaging. A previous \gls{hfd} study observed a type-1 \gls{ifn} response elicited by \glspl{fa} in hepatocytes and macrophages \textbf{\cite{https://gut.bmj.com/content/67/1/157.short}}. Furthermore, the same study noted that murine adipose tissue specific deletion of type-1 \gls{ifn}-receptor 1 (\textit{Ifnar1}) led to metabolic dysregulation, indicated by increased weight gain, insulin resistance and impaired glucose tolerance \textbf{\cite{}}. This suggests a protective role by type-1 \gls{ifn} signaling in adipocytes. Further studies are needed to fully elucidate the role of \gls{ifn} signaling and the downstream signaling components in the development and progression of \gls{t2d} and unravel its complex pathophysiology and advance therapeutic interventions to mitigate its impact.\\

\par At this point, a question still remains as to what triggers the onset of the type-1 \gls{ifn} response in islet macrophages? A plausible mechanism of action involves endogenous danger signals arising from excess levels of nutrients from metabolic overload during obesity. These danger signals, also known as danger-associated molecular patterns (DAMPs), are derived from tissue damage or cellular stress, inclusive of metabolic stress. Metabolism-derived DAMPs such as excess \glspl{ffa}, glucose, cholesterol, glycation end products, oxidized low-density lipoproteins, also known as MAMPs, can instigate the activation of inflammatory processes leading to chronic metabolic disorders \textbf{\cite{https://doi.org/10.1016/j.tem.2020.07.001}}. These MAMPs are sensed by various immune receptors, in particular pattern recognition receptors (PRRs) such as Toll-like receptors (TLRs), Nucleotide-binding oligomerization domain (NOD)-like receptors (NLRs) and other PRRs, and initiate downstream inflammatory cascades such as \gls{nfkb}, MAPKs, and NLRP3 inflammasome pathways. These cascades drive the production of proinflammatory mediators and contribute to the initiation and progression of metabolic disorders like \gls{t2d}. Further comprehensive investigation into the interplay between DAMPs, PRRs, and metabolic disorders offers promising insights for therapeutic interventions targeting metaflammation-related diseases.\\ 


\subsection{Therapeutic avenues in \gls{t2d}}


 % These molecules, derived from excess nutrients and their metabolites are classified as  In contrast to this, there is lack of comprehensive research on the role of type-1 \gls{ifn} signaling in \gls{t2d} in the context of diet-induced obesity or aging. 
% \gls{ifn}-$\gamma$ is also involved in the disruption of the insulin signaling pathway in lean \gls{t2d} patients \textbf{\cite{}}, and in several other complications of \gls{t2d} \textbf{\cite{https://www.doi.org/10.4103/mj.mj_13_21,https://www.doi.org/10.1007/S10096-007-0395-0,https://www.doi.org/10.4239/WJD.V12.I7.939,https://www.doi.org/10.2147/DMSO.S305511}}.

% contributing to chronic inflammation and metabolic dysregulation, thereby linking it to the onset and progression of \gls{t2d}.   


% Moreover, emerging evidence suggests that IFN-γ disrupts cellular metabolism, including mitochondrial function, and interferes with insulin sensitivity, thus implicating it in the etiology of metabolic disorders.



% Interestingly this \gls{ifn} response is stimulus dependent: where aging elicits a canonical response with the activation of signal transducer and activator of transcription 1 (STAT1), metaflammation involved strong alteration of this process and induces a non-canonical response marked by STAT3 activation and the involvement of alternative pathways such as \gls{mtor} signaling.   \\

% \par  While the heterogeneity of the pancreatic islet macrophages in terms of developmental origins, anatomical positioning as well as different activation status in \gls{t1d} has been noted, similar characterization of islet-associated macrophages under metabolic stress had been lacking. Similar to recent studies describing adipose tissue macrophages under homeostasis and obesity \textbf{\cite{https://pubmed.ncbi.nlm.nih.gov/34381461/,https://www.sciencedirect.com/science/article/pii/S1016847824000360#sec0090}}, we provided a detailed characterization of the heterogeneity of islet macrophages under obesity and aging using a multi-\textit{omics} approach. Interestingly, we uncovered remarkable similarity in the heterogeneity and activation status of islet macrophages between \gls{t1d} \gls{nod} model and the diet-induced obesity and aging model in a \gls{t2d} context in our study. This observation reinforces the role of resident macrophages in maintaining islet homeostasis as well as mediating islet inflammation under \gls{t1d} and \gls{t2d} conditions. Building on this, future studies exploring activation differences between \gls{t1d} and \gls{t2d} could elucidate distinct islet inflammation mechanisms.\\
% our study also investigated the impact of diet-induced obesity on the immune dynamics in pancreas during a time-course of feeding with a carbohydrate and fat rich diet, also termed as \acrfull{wd}. This also allowed for direct comparison of the inflammatory processes elicited by obesity and aging. 


% contributes to the growing body of evidence that demonstrates shared inflammatory mediators and pathways between metabolic stress and aging. Using a , we discovered shared characteristics across overnutrition and aging conditions in islet-associated macrophages, specifically the accumulation of the F4/80\textsuperscript{\textit{low}} Macs-3 macrophage sub-population in the pancreatic tissue and islets as well as the activation of a type-1 \gls{ifn} response. Our findings reinforce the concept that 

% \subsubsection{Interferon signaling in \gls{t2d}}

% \par In the context of diabetes research, much attention has been focused toward the role of type-1 \gls{ifn} signaling in \gls{t1d}, wherein its involvement in driving autoimmune responses is well-documented \textbf{\cite{marroqui_type_2021,newby_type_2017}}. Among the type-1 \glspl{ifn}, the contribution of \gls{ifn}-$\alpha$ to the development and pathogenesis of \gls{t1d} is particularly crucial, leading to overexpression of antigen presentation genes in humans (HLA-I) and mouse (MHC-I) alike, \gls{er} stress and $\beta$-cell apoptosis within pancreatic islets \textbf{\cite{marroqui_interferon-_2017,marro_progression_2017,lombardi_interferon_2018,colli_integrated_2020,jiang_interferon-_2022,coomans_de_brachene_interferons_2024}}. Furthermore, the pro-inflammatory and regulatory roles of type-2 \gls{ifn} signaling via \gls{ifn}-$\gamma$ \gls{t1d} has also been noted \textbf{\cite{coomans_de_brachene_interferons_2024,de_george_inflammation_2023}}. 




% The genomes of any two unrelated people are 99.9\% identical.
% Yet, the 0.1\% that differs is critical: it explains why individuals look different, and also why some are more predisposed than others to certain diseases. 
% Thus, identifying DNA variants that are associated with complex disorders, and understanding the molecular mechanisms that mediate such associations, can lead in the future to better disease diagnosis, treatment and prevention.
% While GWAS (\textbf{section \ref{sec:gwas}}) have identified thousands of associations between genetic variants and traits and diseases, the mechanisms involved have proven hard to disentangle.
% Associations between genetic variants and gene expression levels (i.e. eQTL, \textbf{section 
% \ref{sec:eqtl}}) can help uncover such mechanisms, as gene expression often acts as an intermediate between DNA sequence and organismal phenotypes.
% Importantly, since these regulatory effects often arise in specific tissues or under specific stimuli \cite{alasoo2018shared}, eQTL mapping studies need to be conducted in disease-relevant cell types. 
% These are often hard to access, historically limiting studies to easily accessible tissues such as skin and blood \cite{fairfax2014innate, westra2014genome}, or to cell lines \cite{gibson2005quantitative}.
% More recently, the GTEx consortium released eQTL maps across over 50 human post-mortem tissues \cite{aguet2019gtex}.
% Whilst this represents a great resource, these tissues have been probed using bulk RNA-seq, making it difficult to isolate specific disease-relevant cell types, especially since these are often rare.
% Moreover, very little is known about the genetic regulation of gene expression at early stages of human development, most of which are impossible to access \textit{in vivo}.
% Human iPSCs have proven to be a versatile \textit{in vitro} model to study early development in a neatly controlled setup (\textbf{section \ref{sec:ipsc}}).
% Human iPSCs can be derived in a donor-specific manner, and, critically, they can be differentiated towards virtually any cell type of interest.
% Recently, large cohorts of human iPSCs across hundreds of individuals have enabled eQTL analyses in both iPSCs and a number of iPSC-derived cell types \cite{kilpinen2017common, schwartzentruber2018molecular}. 

% In this thesis, I have shown that human iPSC technology combined with single cell expression readouts (which allow the isolation of cell types of interest), and pooling strategies (which increase throughput by enabling the differentiation of cells from several individuals in the same experiment), represent an excellent system to study the effect of common genetic variants on gene expression during cellular differentiation. \\

% In particular, I have analysed two population-scale scRNA-seq datasets of differentiating human iPSCs along two different lineages, one toward definitive endoderm and the other to a midbrain neuronal fate. 
% These represent important resources in their own right, as most current human scRNA-seq datasets only contain samples for a handful of genetically unique individuals. \\

% Indeed, while the main objective of these studies was to identify eQTL across cell types and states during differentiation, one interesting side product of this work was the evaluation of differences in terms of differentiation outcome across several iPSC lines.
% In particular, full transcriptome information across hundreds of iPSC lines allowed us to assay these differences at a much larger scale than any previous study, to the best of my knowledge.
% In one case, we identified a set of genes whose expression at pluripotent stage can be used to predict neuronal differentiation efficiency, and which we could use to predict differentiation scores for the entire HipSci bank.
% This represents important progress toward understanding predictors of differentiation outcome and a useful resource for future studies using these lines. \\

% Nevertheless, the main contributions of this thesis are in the context of eQTL mapping, specifically when using single cell RNA-seq profiles to measure gene expression.
% In particular, we systematically evaluated differences between mapping eQTL using bulk and single cell RNA-seq for a homogeneous cell population (human iPSCs).
% Additionally, we provide preliminary best-practice guidelines for single cell eQTL studies, in terms of normalisation strategies, aggregation approaches and covariate adjustment.\\

% Furthermore, we mapped eQTL at different stages and cell types along human early development toward endoderm (mesendoderm and definitive endoderm) and along the midbrain neural lineage (floor plate progenitors, dopaminergic and serotonergic neurons, ependymal cells and astrocytes). 
% To the best of our knowledge, these are the first eQTL maps at these stages of differentiation and thus represent an important resource for the genetics community. \\

% Finally, work in this thesis provides insight into the importance of performing genetic analyses of gene expression in a context-specific manner, both by performing eQTL in discrete cell types and stimulation states, and by considering continuous axes of variation which modulate the genetic response.

% \section{Conclusions and discussion}

% The analyses we conducted have several important implications, in two main areas, which I discuss in the following sections.
% First, I use \textbf{section \ref{sec:discussion_part1}} to summarise and discuss our results assessing variability in the differentiation outcome of human iPSC lines and possible molecular predictors.
% Second, in \textbf{section \ref{sec:discussion_part2}} I discuss technical considerations and biological implications of mapping eQTL using single cell expression profiles.

% \subsection{Human iPSCs to model development and disease}
% \label{sec:discussion_part1}

% In work presented in this thesis, we attempted to quantify the differentiation efficiency of different iPSC lines in two distinct protocols.
% In the first case, (described in \textbf{Chapter 
% \ref{chapter4}}) the protocol used was very short (three days) and very well understood, describing early stages of endoderm differentiation.
% Even so, we observed noticeable differences between lines in their ability to differentiate towards definitive endoderm.
% We identified a few tens of genes whose expression at iPSC stage was predictive of endoderm differentiation efficiency (\textbf{section \ref{sec:endodiff_differentiation_efficiency}}).
% These were mostly on chromosome X, confirming previous reports that the X chromosome reactivation in human iPSC lines may hamper their quality, especially with regards to their differentiation potential.\\

% In the second study I describe in this thesis (in \textbf{Chapter \ref{chapter5}}) the differentiation protocol used was much longer (52 days), and we differentiated significantly more lines (215).
% Here, we observed even more extreme differences across lines in their ability to generate neurons, with roughly one third of the lines preferentially producing non-neuronal cell types, namely ependymal- and astrocyte-like cells.
% Similar to previous reports \cite{schwartzentruber2018molecular}, some batch effects were observed, but were significantly weaker than cell line effects.
% On the other hand, we identified an iPSC gene signature that was predictive of poor neuronal differentiation efficiency, finding around two thousand genes whose expression at pluripotent stage was significantly correlated (either positively or negatively) with a line's ability to generate neurons. 
% We further hypothesised that this may be linked to a sub-population of iPSCs that exhibited differential expression of these genes.
% We speculate on possible mechanisms (\textbf{section
% \ref{sec:neuroseq_discussion}}), but argue that further validation would be needed to state anything conclusively.
% Lastly, we observe no correlation between the differentiation efficiencies defined in the two protocols, suggesting that a line's differentiation potential toward one lineage is independent, or perhaps even inversely correlated with that toward another. \\

% Future work is required to gain a better understanding of the mechanisms and causes behind an iPSC line's differentiation potential.
% In particular, we note that since in both studies we only chose to differentiate one cell line per individual, we could not distinguish between cell line effects and donor effects.
% In future efforts, it will be important to include multiple lines per individual, to be able to effectively separate the two sources of variation.
% Moreover, all lines used here are skin-derived, thus the differences observed could not be driven by the somatic cell type of origin.
% A future area of study would involve investigating differences in the differentiation outcome of iPSC lines derived from different cell types, as well as across donor characteristics including sex, age and ethnicity.
% As highlighted on \textbf{page \pageref{sec:HipSci}}, several human iPSC cohorts derived from different cell types, and for donors of different ethnicities and varying degrees of relatedness, are already available for research purposes, and could be used to address some of these aspects.
% Finally, in work presented here we did not have the appropriate sample size to detect genetic variants affecting differentiation efficiency.
% In the future, as protocols become more efficient and pooling strategies combined with single cell readouts become common-practice, it will be possible to perform \textit{in vitro} differentiation experiments at increasingly large sample sizes.
% These studies will finally enable the exploration of the potential role of genetic variation on differentiation outcomes. \\

% As more and more \textit{in vitro} differentiation studies are conducted, across different lineages and iPSC cohorts, a systematic comparison of the outcomes can be performed, which will greatly improve our understanding of the processes involved.
% Indeed, such comparative studies will shed light on several unanswered questions.
% For example, is an iPSC line's inability to differentiate toward mature cell types simply an indication of its poor quality?
% And if so, is it simply not possible to use these lines for differentiation studies?
% Or, alternatively, are some lines more prone toward one cell fate and as a consequence less so to another?
% And to what extent is this dependent on the cell type of origin of those iPSCs?
% Importantly, is poor differentiation ability a characteristic of the cell line, or of the donor (genetic or otherwise)?
% This would have critical consequences, for example on the importance on deriving several iPSC lines from the same donor to maximise yield of `good differentiating lines'.
% And if not, will it be harder to derive functional iPSC lines from some individuals compared to others?
% These and other questions remain to be investigated.

% \subsection{Bridging the genotype-phenotype gap}
% \label{sec:discussion_part2}

% A large gap remains in our understanding of the functional mechanisms that link genotypes to phenotypes.
% eQTL studies can be used to fill some of this gap by identifying the putative regulatory role of common variants on gene expression.
% Indeed, when performed across tissues and contexts, eQTL maps can provide insights not only into which genes are regulated, but also in which cell types and under which conditions they are active.\\

% The profiling of molecular traits, especially gene expression, at single cell resolution has represented a true revolution in the last ten years (\textbf{section \ref{sec:scrnaseq}}).
% In particular, experimental methods, and computational approaches to examine the resulting data, have become established in recent years, leading to the explosion of scRNA-seq data, with > 1,000 datasets published since 2009.
% Single cell expression profiling can now be deployed at population-scale and, combined with pooling strategies, permits the efficient quantification of cell-level expression across several individuals.
% Additionally, single cell transcriptomics can be used to estimate cell states and contexts at increased resolution \cite{buettner2017f}.
% For example, rare cell types and cells in different cell cycle phases can be identified unbiasedly within one experiment.
% Lastly, the use of single cell expression profiles allows the ordering of single cells along a continuous trajectory, without the need to discretise cells into distinct populations.
% Adding such cell-level context information to eQTL mapping provides one more layer to our understanding of the molecular consequences of common genetic variation, potentially making the genotype-phenotype gap one bit smaller.\\

% In this thesis, I provide examples of how the single cell resolution of expression profiles can be leveraged to better understand the molecular machinery of gene regulation.
% First, single cell expression profiles can be used to unbiasedly identify pure cell populations, quantify expression within those, and then test for eQTL in such populations.
% Second, single cell profiles can be used to order cells along a differentiation trajectory, and used to identify dynamic eQTL, i.e. eQTL whose strength varies over time.
% Third, single cell transcriptomic data can be used to define other axes of variation, and thus context-specific eQTL can be identified across a plethora of cell states.
% From a technical standpoint, linear and linear mixed models (\textbf{Chapter \ref{chapter2}}) are flexible frameworks that allow the user to efficiently test for associations whilst correcting for confounders and other sources of variation.
% Finally, colocalisation analysis between the identified eQTL and relevant GWAS trait connects the final dots to link the identified regulatory mechanisms to complex traits and diseases. \\

% Historically, eQTL have been mapped using bulk RNA-seq profiles as a measure of expression level.
% The first implication of work described here is the feasibility of large-scale genetics using single cell RNA-seq data instead.
% Whilst this has to an extent been demonstrated before \cite{van2018single, kang2018multiplexed}, the small sample size of those studies only allowed the identification of tens or at most a few hundred eQTL.
% Moreover, these studies failed to recapitulate eQTL results obtained using bulk RNA-seq from equivalent tissues.
% Here, on the other hand, we identify thousands of eQTL across a range of cell types (\textbf{Tables \ref{tab:endodiff_eqtl_summary}, \ref{tab:eqtl_results}}), and could re-discover a larger portion of bulk-discovered eQTL (\textbf{Fig. \ref{fig:sc_bulk_egenes}, \ref{fig:neuroseq_and_gtex_rediscovery}}). 
% Moreover, we demonstrated feasibility of single cell eQTL mapping using both plate-based (SmartSeq2, \textbf{Chapters \ref{chapter3}, \ref{chapter4}}) and droplet-based (10X Genomics, \textbf{Chapter \ref{chapter5}}) scRNA-seq data. \\

% To systematically compare the performance of using scRNA-seq as opposed to bulk RNA-seq to map eQTL, in \textbf{Chapter \ref{chapter3}} we selected human iPSCs as a homogeneous cell type, and compared results when mapping iPSC eQTL using a common set of samples (\textbf{Fig. \ref{fig:sc_bulk_egenes}}). 
% This analysis revealed an increased number of discovered eQTL when using bulk RNA-seq profiles in this well defined, pure cell population, probably due to decreased noise in expression estimates.
% On the other hand, we appreciated the power of the single cell transcriptomics to isolate several cell types within more heterogeneous populations, quantify expression and map eQTL within them, without the need for any gating or other experimental techniques to separate cell populations (\textbf{Fig. \ref{fig:endodiff_stage_eqtl}, \ref{fig:neuroseq_eqtl_examples}}). \\

% In addition, we provide here the first hints towards the establishment of a best-practice workflow to maximise yield of single cell eQTL studies (\textbf{section \ref{sec:best_practice}}), identifying the mean (after single cell-specific normalisation) as the optimal aggregation method, and principal component analysis as the preferable approach to capture global expression trends which should be included in the model as covariates. 
% From a methodological perspective, linear mixed models were confirmed as the appropriate tool to identify genetic associations, given their ability to deal with confounding effects (\textbf{section \ref{sec:confounders}}).
% In particular, LMMs can control for effects due to population structure, including replicate measurements across donors (for example across multiple differentiation experiments, \textbf{Chapters \ref{chapter3}} and \textbf{\ref{chapter4}}).
% In addition, LMMs enabled the introduction of a variance term to account for number of cells across individuals, which varied widely thus rendering the expression estimates less precise (\textbf{Chapter \ref{chapter5}}); this expedient resulted in a great boost in the number of eQTL discoveries. 
% In future work, models which enable the incorporation of multiple random effect terms, to effectively correct for several confounders simultaneously, should be developed. 
% \\

% The availability of eQTL maps across cell types and stages provided the opportunity to assess the amount of eQTL signal sharing both within our studies and in comparison with existing maps.
% This is a notoriously complicated task, because different eQTL studies may differ in the technology used to measure expression, in the number of genes expressed and in sample size, which is in general fairly low.
% Here, we used two separate approaches to tackle this issue.
% On the one hand, we used p value thresholding to identify cell type-specific eQTL (eQTL that could only be detected in one of the cell populations considered within our study, \textbf{Fig. \ref{fig:endodiff_stage_specific_eqtl}, \ref{fig:neuroseq_eqtl_examples}}), and assess the number of eQTL identified in our study that were not discovered in eQTL maps of primary tissues (i.e. from GTEx) and viceversa (\textbf{Fig. \ref{fig:neuroseq_and_gtex_rediscovery}}).
% On the other hand, we used a recently proposed method (MASHR \cite{urbut2019flexible}) to quantify genome-wide sharing across eQTL maps (\textbf{Fig. \ref{fig:neuroseq_and_gtex_brain_sharing}}).
% These approaches are complementary, representing the two ends of the spectrum: the first approach may miss signals that only just do not reach the (arbitrary) significance threshold used, whilst the second may overestimate sharing by only considering gene-SNP pairs assessed across all conditions included.
% To partially overcome these issues, methods exist that consider multiple eQTL datasets jointly \cite{flutre2013statistical, sul2013effectively}. 
% However, such methods are currently computationally too demanding for large-scale scRNA-seq data. \\

% Next, in \textbf{Chapter \ref{chapter4}}, we added the temporal axis, by identifying dynamic eQTL, i.e. eQTL whose strength is modulated by developmental time.
% This extends similar work from \cite{francesconi2014effects, strober2019dynamic}, to single cell-resolved data.
% Indeed, in this study cells were collected at very close time points, which combined with varying differentiation rates across both cells and lines resulted in a continuous differentiation trajectory.
% Importantly, we observed that changes in genetic effects over time did not merely reflect changes in overall expression (\textbf{Fig. \ref{fig:endodiff_dynamic_eqtl}}).
% Moreover, we found that dynamic eQTL were enriched for epigenetic marks consistent with promoter and enhancer regions.
% We next used the same approach, building on allele-specific expression (similar to \cite{knowles2017allele} for GxE), to test for eQTL effects that are modulated by alternative cell states, including cell cycle phase and metabolic state (\textbf{Fig. \ref{fig:endodiff_gxe}, \ref{eq:endodiff_ase_gxexe}}).
% This type of analysis is similar to previous work to identify `interaction eQTL' \cite{zhernakova2017identification, van2018single}. \\

% Finally, in \textbf{Chapter \ref{chapter5}}, we assessed disease-relevance of our identified associations, by performing colocalisation analysis between eQTL maps from our neuronal cell populations and GWAS for neurological traits.
% Here, we uncover several colocalisation events that had not been previously identified (\textbf{Fig. \ref{fig:neuroseq_coloc_overview}}), highlighting once again the importance of studying the molecular consequences of genetic variation in relevant cell types, especially when investigating the genetic basis of disease.
% Indeed, some of these examples provide insight into the genetic underpinning of neurological diseases, including schizophrenia.

% Overall, the work in this thesis demonstrates the feasibility of eQTL mapping using single cell expression profiles and the importance of modelling context-specific eQTL effects across cellular types and states.
% The methods used build on the linear mixed model framework and are extremely flexible, as demonstrated by their application across technologies and designs. 
% Whilst extremely useful and efficient, these models assume normality of the residual phenotypes, an assumption that is often violated, as discussed (\textbf{section \ref{sec:non_gaussian}}).
% In particular, scRNA-seq data has has been described to follow a Poisson or a negative binomial \cite{grun2014validation, hafemeister2019normalization, svensson2020droplet} distribution.
% Future work should include the evaluation of the feasibility of integrating non-Gaussian likelihoods in the models to map eQTL using scRNA-seq data. \\

% Moreover, in this thesis I have focused on the study of context-specific eQTL by either first discretising cells into populations and mapping eQTL in each, or by considering interactions with one single continuous cell state (or at most two, \textbf{Fig. 
% \ref{fig:endodiff_gxexe}}).
% In the future, it will be important to develop methods to jointly test for context-specificity of eQTL across several (continuous and discrete) cell states simultaneously.
% For example, a recently proposed method, Struct-LMM \cite{moore2019linear} allows the assessment of GxE interactions using larger numbers of conditions.
% While originally proposed in the context of population studies, the same principles could be adopted here, where one could map sc-eQTL that vary jointly across up to hundreds different cell states and types. 
% These advanced models will enable us to leverage the rich information from single cell-resolved, transcriptome-wide population-scale datasets, to further improve our understanding of the genetic architecture of traits.


% \section{Outlook and future directions}

% \subsection{More complex and realistic \textit{in vitro} models}

% Work presented in this thesis demonstrated how iPSC differentiation combined with multiplexed experimental designs and single cell RNA-seq profiling unlocks population-level studies in increasingly complex, dynamic and biologically realistic cellular models. 
% We anticipate that, in the future, uses of this model system will focus on experimental settings that are challenging or impossible with primary cells. 
% For example, these may include single cell resolution sampling along longer differentiation times to more complex differentiation trajectories, such as cell organoids, or involve large panels of disease relevant-stimuli and drug exposures. 
% These future efforts will greatly contribute to our understanding of the common genetic basis of complex disorders, and facilitate the development of iPSC-based approaches for modelling, and even eventually treating, these diseases. 

% \subsection{More population-scale scRNA-seq datasets}

% The work presented in this thesis has focused on applications in iPSCs and iPSC-derived cells, however we note that the same technologies and methodologies can be applied across a variety of biological systems.
% For instance, scRNA-seq \gls{pbmc} data across multiple conditions from a growing number of individuals (currently approximately 1,600), will be available as part of the single-cell eQTLGen (sc-eQTLGen) consortium, whose manifesto was recently published \cite{van2020single1}.
% Similarly, through the UK Biobank \cite{bycroft2018uk}, one of the largest and most deeply phenotyped cohorts of individuals in the world, blood samples as well as key biomarkers for circa 500,000 people are stored and available for the research community.
% Performing scRNA-seq on blood cells from these many (and well characterised) samples will provide an invaluable resource to study the effect of genetic variants across cell types and contexts.
% Last but not least, the human cell atlas (HCA) project \cite{regev2017science}, whose mission is “to create comprehensive reference maps of all human cells”, will likely in the future be collecting samples from several donors across all human tissues, to evaluate differences across genetic backgrounds and disease states.
% It will be critical, when these data become available, to have robust and efficient statistical models to make use of this wealth of data.

% \subsection{Alternative single cell technologies}

% Finally, here we have focused on single cell transcriptomic data, which is the most well-established of the single cell sequencing technologies.
% Yet more recently, several alternative molecular traits have been assayed at single cell resolution, including chromatin accessibility  \cite{buenrostro2018integrated, corces2016lineage}, DNA methylation  \cite{guo2013single, smallwood2014single, farlik2015single}, histone modifications \cite{rotem2015single} and chromatin 3D organisation \cite{nagano2013single}.
% Novel technologies even allow multiple molecular layers to be probed in parallel from the same individual cells \cite{stoeckius2017simultaneous, cao2018joint, clark2018scnmt}.
% The LMM-based models used here can readily be adapted to map alternative single cell molecular QTL (e.g. sc-mQTL, sc-caQTL, etc.), which could provide a much richer understanding of the molecular machinery associated with genetic regulation.
% Using multi-omics data, similar models can further be used to study the interplay between molecular layers (e.g. effects of methylation or accessibility on expression).
% Finally, standard eQTL assess the effect of naturally occurring genetic variation on gene expression.
% However, recent pioneering studies have used the induction of CRISPR/Cas9 perturbations, followed by scRNA-seq to identify the effect of such induced variation on gene expression \cite{gasperini2019genome}.
% Models describe here can naturally be extended for the identification of cell type- and context-specific `crisprQTL' as well.

% \section{Genetic mapping at single cell resolution}

% The use of single cell omics, particularly gene expression, has revolutionised our understanding of cellular variability in several biological systems.
% These technologies can now be deployed across hundreds of individuals, enabling the study of the effects of common genetic variants on gene expression level (i.e. by mapping eQTL), which was once only assessed using bulk RNA-seq.
% In particular, the single-cell resolution can help to uncover eQTL that are only active in rare cell populations, or that change dynamically along cellular states. 
% Taken together, the work in this thesis demonstrates the utility of mapping eQTL using single cell expression data, to reveal the function of genetic variation across cellular types and states.
% The models described here, combined with the increasing availability of population-scale single cell expression studies, and in the future extended to include multiple molecular layers, have the potential to greatly advance our understanding of the complex machinery that links genotype to phenotype. \\

\section{Shortcomings}

\section{Concluding Remarks}